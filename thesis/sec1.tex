%%%%%%%%%%%%%%%%%%%%%%%%%%%%%%%%%%%%%%%%%%%%%%%%%%%%%%%%%%%%%%%%%%%%
% Einleitung
%%%%%%%%%%%%%%%%%%%%%%%%%%%%%%%%%%%%%%%%%%%%%%%%%%%%%%%%%%%%%%%%%%%%

\chapter{Introduction}\label{Introduction}

A good introduction is of utmost importance.


\medskip
At the end of the introduction follows a text similar to the following:

The paper is structured as follows: The basics of ...
are developed in chapter~\ref{Introduction}. 
...
A
discussion and a brief outlook in the
chapter~\ref{Discussion} conclude this work.

Before we turn to the evaluation and assessment of the primary data obtained, we would first like to repeat some basic concepts of descriptive statistics.

\section{Random Samples}
In principle, with a sample of RNA-seq data \cite{springer} we are dealing with a {\em sample} from a {\em population}.   
We now generally denote the observed data of size $n$ with $X=\{x_1,x_2,\ldots,x_n\}$. 
This data should be described with statistical parameters. From these, we want to infer the underlying distribution in the population as reliably as possible. For this purpose, we use the {\bf location} and {\bf variance parameters}. First, however, we turn to the frequency and cumulative frequency distributions, which provide an impression of the distribution of $X$ both graphically and numerically. For this purpose, we consider discrete distributions.

Given a sample $(X_1,X_2,\ldots,X_n)$. A function $Z_n=Z(X_1,\ldots,X_n)$ is called a {\em sampling function}. It is itself a random function.


\subsection{Frequencies and Histograms}
In $X$ the value $x_i$ occurs exactly $n_i$ times, $i=1,2,\ldots m$. Then $\sum_i n_i = n$. The quotient $n_i/n$ is the {\em relative frequency} for the occurrence of the event ``$X=x_i$''.
The set of relative frequencies $\{n_1/n,n_2/n,\ldots, n_m/n\}$ is {\em frequency distribution} of $X$. Furthermore, the set $\{s_1,\ldots,s_m\}$ with $s_i=\sum_{k=1}^{i}n_k/n$ is the {\em total frequency distribution} of $X$.

For the graphical representation of the frequency distribution, the {\em histogram} (see Fig.~XX) is selected. for the cumulative frequency distribution, the {\em staircase function}.

%
\subsection{Important Distributions}

\subsubsection{The Normal Distribution}


